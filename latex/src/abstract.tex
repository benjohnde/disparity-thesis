Stereoskopische Videos speichern zwei separate Ansichten einer Szene, die sich {\"u}blicherweise nur in geringen horizontalen Verschiebungen der Pixel unterscheiden. Diese Pixelverschiebungen (Disparity) resultieren aus den unterschiedlichen Entfernungen der Objekte in der Szene. Ziel der Master-Abschlussarbeit ist es, bestehende Verfahren zur Berechnung der Disparity zu analysieren, geeignete Verfahren f{\"u}r Videos auszuw{\"a}hlen und diese zu implementieren. Ein spezieller Fokus soll auf Erweiterungen f{\"u}r Videos liegen, indem beispielsweise vorherige oder zuk{\"u}nftige Frames ber{\"u}cksichtigt werden. Zur Messung der Qualit{\"a}t der implementierten Verfahren sollen bestehende stereoskopische Bild- und Videoarchive genutzt werden.
