\chapter{Conclusion}
\label{chap:conclusion}

The work described in this thesis has been concerned with the comparison of disparity algorithms.
For this purpose, evaluation methods were introduced and integrated.
As stereoscopic datasets focusing on videos are rarely spread, the ones, which were used in the evaluation, were also introduced.
Additionally, a simple stereo matcher focusing on videos was implemented described.
The following section recaps this in more detail and furthermore a outlook and future work are given.

\section{Thesis summary}

The thesis brought foundations, describing 

\begin{itemize}
  \item What the thesis brought with each chapter.
  \item One small paragraph regarding one disparity algorithm.
  \item One small paragraph regarding dataset of the chair.
  \item What the main result of the evaluation.
\end{itemize}

\noindent The following modules were actually implemented:

\begin{itemize}
  \item Reader for the PFM file format.
  \item Python scripts for evaluation.
  \item Shell scripts for creating docker containers and distribute the work among different server instances.
  \item Evaluation processor based on OpenCV for stereoscopic videos compared with their ground-truth companion.
  \item A simple stereo matcher targeting videos by holding a spatiotemporal context.
  \item An image diminisher utilizing FFmpeg to simulate noise and video compression artifacts.
\end{itemize}

\section{Outlook and future work}

With the web viewer, a tool for sharing benchmark results was presented.
This tool could be enhanced even more to visualize the results with graphs.
Another possibility is to let researcher submit their dataset and the whole eval-chain is then executed on a server with the results being shown afterwards.
The simplest idea to enhance the evaluation engine is, to add more metrics and more algorithms.
\newline\newline\noindent With the simple stereo matcher, a skeleton for creating a four-dimensional disparity space image was presented.
This could be a starting point for further research to optimize stereoscopic videos respecting the spatiotemporal context.
Concluding, the SNSM performed reasonable but seemed to be a bit random in some cases.
Thus, the next steps could be to try to focus on non-moving objects or to track the movement of objects via Optical flow to take only static parts into account.
\newline\newline\noindent Future work regarding enhancement of disparity algorithms in general are for instance to implement other matching cost calculation methods and evaluate those \citep{hermann2010gradient}.
Another approach could be to focus more on how humans experience depth \citep{deangelis1995neuronal} and to combine those learnings with neuronal networks \citep{olshausen1996emergence} which has not been done yet.
\newline\newline\noindent In general, the available datasets lack of high resolutions.
Also multi-view datasets are not available with high-resolution images at all.
Although some real-world ground-truth disparity maps exist, they normally lack of accuracy and tend to be available in only low resolutions \citep{Geiger2011IV}.
One approach towards the direction of high resolution real-world stereoscopic videos was made by \citeauthor{kondermann2015stereo} \citep{kondermann2015stereo}.
They provide a error-bar how accurate the sensed disparity is.
As this is currently the only dataset of that nature, there is still room for more.
\newline\newline\noindent As a more general outlook, algorithms like ELAS \citep{Geiger2010ACCV} are more demanded due to low runtime and the accuracy in high-resolution images for real-time applications like autonomous driving.
