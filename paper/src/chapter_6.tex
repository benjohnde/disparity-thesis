\chapter{Conclusion}
\label{chap:conclusion}

The work described in this thesis has been concerned with the comparison of disparity algorithms.
For this purpose, evaluation methods were introduced and integrated.
As stereoscopic datasets focusing on videos are rarely spread, the ones, which were used in the evaluation, were also introduced.
Additionally, a simple stereo matcher focusing on videos was implemented described.
The following section recaps this in more detail and furthermore a outlook and future work are given.

\section{Thesis summary}

\begin{itemize}
  \item What the thesis brought with each chapter.
  \item One small paragraph regarding one disparity algorithm.
  \item One small paragraph regarding dataset of the chair.
  \item What the main result of the evaluation.
\end{itemize}

\noindent The following modules were actually implemented:

\begin{itemize}
  \item Reader for the PFM file format.
  \item Python scripts for evaluation.
  \item Shell scripts for creating docker containers and distribute the work among different server instances.
  \item Evaluation processor based on OpenCV for stereoscopic videos compared with their ground-truth companion.
  \item A simple stereo matcher targeting videos by holding a spatiotemporal context.
  \item An image diminisher utilizing FFmpeg to simulate noise and video compression artifacts.
\end{itemize}

\noindent Summarize the results of the evaluation.

%todo add recommendations for svddd dataset

\section{Outlook and future work}





was man noch verbessern kann
- simple stereo matcher
- web viewer
- darstellen, dass der SNSM ganz gut funktioniert hat
- ELAS als überraschungskandidaten
- auf die anderen vorhandenen stichpunkte bisschen eingehen





A few thoughts on possible future work are outlined.
The thoughts are split in low- and high-level, from a technical perspective.
Low-level items are focused towards underlying methods of disparity algorithms, whereas high-level items focus on the bigger picture and tools which may help in developing stereo matcher.

\begin{itemize}
  \item Neuronal networks \citep{olshausen1996emergence}
  \item Other matching cost calculation methods \citep{hermann2010gradient}.
  \item Focus more on how humans experience depth? \citep{deangelis1995neuronal}
\end{itemize}

\begin{itemize}
  \item Real-time availability
  \item higher resolution
  \item What's on the market, like multi-view?
  \item Provide more real-world ground truth disparity maps \citep{kondermann2015stereo, Geiger2011IV}.
\end{itemize}
